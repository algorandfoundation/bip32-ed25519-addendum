\title{Bip32-Ed25519++: A More Secure Hierarchical Deterministic Wallet Standard for Ed25519 Keys}

% \author{Chris Peikert\cite{ChrisP}}

\newcommand{\abstractText}{\noindent

}

%%%%%%%%%%%%%%%%%
% Configuration %
%%%%%%%%%%%%%%%%%

\documentclass[12pt, a4paper, twocolumn]{article}
\usepackage{microtype}
\usepackage{xurl}
\usepackage[sort&compress]{natbib}
\usepackage{abstract}
\renewcommand{\abstractnamefont}{\normalfont\bfseries}
\renewcommand{\abstracttextfont}{\normalfont\small\itshape}
\usepackage{amsmath} % Add this line to define the \text command
\usepackage{titlesec}
\usepackage[skip=10pt]{parskip}
\usepackage{enumitem}

\titleformat*{\section}{\large\bfseries}
\titleformat*{\subsection}{\normalsize\bfseries}

% Any configuration that should be done before the end of the preamble:
\usepackage{hyperref}
\hypersetup{colorlinks=true, urlcolor=blue, linkcolor=blue, citecolor=blue}
\usepackage[capitalize,nameinlink,noabbrev]{cleveref} % must load after hyperref

\begin{document}

%%%%%%%%%%%%
% Abstract %
%%%%%%%%%%%%

\twocolumn[
  \begin{@twocolumnfalse}
    \maketitle
    \begin{abstract}
      \abstractText
      Implementing BIP32-style HD wallets for Ed25519 keys is a non-trivial task due to the non-linear key space of Ed25519. The paper BIP32-Ed25519 Hierarchical Deterministic Keys over a Non-linear Keyspace has seen adoption in the industry, but has some security issues. In this paper modifications are proposed that addresses those issues, at the cost of reducing the number of possible levels of child keys.
      \newline
      \newline
    \end{abstract}
  \end{@twocolumnfalse}
]

%%%%%%%%%%%
% Article %
%%%%%%%%%%%


\section{Introduction}

Blockchains rely on users to generate seed phrases that correspond to public/private keypairs for use in signing transactions on behalf of an address.

In the basic case, a single seed phrase corresponds to a single address, requiring users to keep track of many seed phrases should they wish to use separate addresses for different application domains: making and receiving payments, holding NFTs, participating in a DAO, DeFi, gaming, etc.


\subsection{Hierarchical Deterministic Wallets}
\label{subsec:hd_wallets}

Bitcoin Improvement Proposal 32 (BIP32~\cite{BIP32}) was introduced to provide users with a means to deterministically derive many public/private keypairs from a single secret seed, greatly easing the management of seed phrases.

BIP43~\cite{BIP43} introduced the concept of a purpose field, claiming that the original BIP32 standard allowed for too many degrees of freedom and resulted in many wallet softwares which claimed to be BIP32 compatible but were ultimately incompatible with each other. BIP44~\cite{BIP44} took this a step further and defined 5 levels in the BIP32 path:

$m / \text{purpose}' / \text{coin\_type}' / \text{account}' / \text{change} / \text{key}$

\begin{itemize}[label=\textendash, itemsep=-0.5em]
  \item m: master seed
  \item purpose': derivation path scheme (e.g., 44')
  \item coin\_type': constant corresponding to the blockchain
  \item account': user account
  \item change: 0 for external addresses, 1 for internal addresses
  \item key: address index
\end{itemize}

The apostrophe in the paths indicates that the respective field is hardened, meaning that the child key is derived in a way that prevents the child public key from being derived publicly from the parent public key, or the sibling keys of that child key. The derivation method is different for hardened keys, requiring the parent private key.

Each level has $2^{32}$ possible values. When hardening a level, the first bit is set to 1. Thus $[0, 2^{31})$ are reserved for non-hardened keys, and $[2^{31}, 2^{32})$ are reserved for hardened keys.

The terms ``account'' and ``address'' are sometimes used interchangeably in casual conversation. In the HD wallet context however, an address is the most basic unit, while an account represents a logical grouping of many addresses. 

Change is used to differentiate between addresses used for receiving payments and addresses used for change in transactions, but this is more relevant for UTXO blockchains. Account-based blockchains like Algorand or Ethereum do not have this distinction, and can either set it to 0 or reserve it for other use.

Since the indices and change levels are not hardened but the account level is, it is possible to connect sibling addresses within an account by providing the extended public key produced at the account level. However, different accounts cannot be connected with each other should the extended public key at the coin\_type level be leaked. Similarly, hardening at the coin\_type level means that keys derived for one blockchain cannot be connected to keys derived for another blockchain.

This class of wallets is known as Hierarchical Deterministic Wallets, or HD Wallets for short.

Other derivation schemes exist as well, such as BIP49~\cite{BIP49} for P2SH nested SegWit addresses, and BIP84~\cite{BIP84} for native SegWit addresses. ZIP32~\cite{ZIP32}, a variant used by ZCash for their shielded addresses, does away with the change field completely.

A complementary set of standards called SatoshiLab Improvement Protocols (SLIP) also exist, maintained by SatoshiLab (the creators of the Trezor hardware wallet) and come up in the context of BIP32 wallets. For example, SLIP10~\cite{SLIP10} or SLIP23~\cite{SLIP23}. SLIP44~\cite{BIP44} extended BIP44 to assign $\text{coin\_type}$ values for blockchains other than Bitcoin mainnet and testnet, e.g. $\text{283}'$ for Algorand.

Note that all of the schemes listed above rely on 5 or fewer levels of derivation.

\subsection{Elliptic curves and their key spaces}
\label{subsec:elliptic_curves_and_their_key_spaces}

Bitcoin relies on the Secp256k1~\cite{Secp256k1} elliptic curve, which has a linear key space. A new public key can be formed by adding $[x]B$ to the public key $[k]B$, where $x$ and $k$ are private keys/scalars and $B$ the base point. This results in a new public key whose private key is $[x] + [k] \text{ mod } q$ for $q$ the order of the field.

An alternate elliptic curve standard for digital signatures is Ed25519~\cite{Ed25519}. It is used by a number of blockchains such as Algorand~\cite{AlgorandEd25519}, Cardano~\cite{CardanoEd25519}, Monero~\cite{MoneroEd25519}, Nano~\cite{NanoEd25519}, Solana~\cite{SolanaEd25519} and more. Polkadot~\cite{PolkadotEd25519} uses a variant called Schnorrkel/Ristretto x25519 (``sr25519'') and Tezos~\cite{TezosEd25519} offers it alongside other curves.

Ed25519 has a number of advantages, such as high speed constant-time implementations and a signing procedure where the nonce is deterministically derived from the data to be signed (mitigating the kind of attack that Sony PS3 suffered~\cite{Ed25519}).

Unfortunately, private keys have to be derived by hashing the seed and manipulating the bits. RFC\-8032~\cite{RFC8032} specifies that of the 32 byte private key:

\begin{itemize}[label=\textendash, itemsep=-0.5em]
  \item Bits 0, 1, 2:  set to 0, ensuring that the key is a multiple of 8 and avoiding small subgroup attacks.
  \item Bit 254: set to 1, to ensuring the Montgommery ladder computation used in scalar multiplication runs in constant time.
  \item Bit 255: set to 0, ensuring the key is in the range of the field order ($2^{255} - 19$).
\end{itemize}

By definition the key space for Ed25519 is non-linear.

As a consequence of this fact, the SLIP10~\cite{SLIP10} standard decided to not support public key child derivation for Ed25519 keys. Only hardened key generation from private parent key to private child key is supported by it. This is, as of the time of writing, what Solana supports~\cite{SolanaBip32}. 

In contrast, a paper that has seen recognition in the industry is BIP32-Ed25519
Hierarchical Deterministic Keys over a Non-linear Keyspace~\cite{BIP32-Ed25519}. It extends the BIP32 standard to work with Ed25519 keys with public key child derivation.

It has been adopted by Cardano, though conforming additionally with SLIP23~\cite{SLIP23} to derive the initial master Ed25519 key and chain code.

While solving the problem (with only minor modifications to BIP32 and working with ``after-hash'' private keys of Ed25519)~\cite{BIP32-Ed25519}, certain decisions were made that result in unnecessary security degradation.

In this paper an improvement to their method is proposed that provides a more secure way to derive Ed25519 keys, at the cost of reducing the number of levels of child keys.


\section{Security Review}

While reviewing the BIP32-Ed25519 paper~\cite{BIP32-Ed25519}, two issues were identified, the first minor and the other major.

\subsection{Discarding bad child private or public keys}

While deriving a new private or public child key, it is possible to produce a child that is unsafe to use:

\begin{itemize}[label=\textendash, itemsep=-0.5em]
  \item $k_L$ is divisible by the base order $n$
  \item $A_i$ is the identity point (0, 1)
\end{itemize}

The paper states that if these happen the child must be discarded. However it does not give instructions on how to proceed with the derivation.

For example, an implementer could mark the index as forbidden and simply increment the child index. However, depending on the level, should the user share their extended public key to the world and others derive the same faulty child key, the security of the parent key would be degraded.

Regardless, the likelihood of either happening is highly unlikely.

\subsection{The 28 left bytes of Z}
\label{subsec:the_28_left_bytes_of_z}

As was mentioned in \cref{subsec:elliptic_curves_and_their_key_spaces}, there is a need to ensure that bits 0, 1, 2, 3 and 255 are 0 and that bit 254 is 1 for private keys (scalars) in Ed25519.

The paper's method for ensuring this is to produce the child private key respectively public key as follows:

Derive $Z$ from either the parent's public key or parent's extended private key, depending on whether the child is hardened or not:

$  Z \leftarrow 
  \begin{cases} 
  F_{c^P}(0x02||A^P||i) & i \in  [0, 2^{31}) \\
  F_{c^P}(0x00||k^P||i) & i \in [2^{31}, 2^{32})
  \end{cases}
  $

Truncate the 32-byte $Z$ to 28 bytes (224 bits): $Z_L \leftarrow Z_{[:224]}$.

Depending on the choice, calculate the child private key (scalar) or public key (which corresponds to an address on the blockchain):

\begin{enumerate}[label={}] 
  \item $k_L \leftarrow \langle 8[Z_L]  + [k{_L}^P]\rangle$ 
  \item $A_i \leftarrow A^P + [8Z_L]B$
\end{enumerate}


The calculations of $k_R$, which are necessary to produce the extended private key $k = (k_L, k_R)$, and the child chain code $c_i$, which is necessary to serve as the key for the HMAC-512 function $F$ (and be part of the extended public key $(A_i, c_i)$); have been omitted for brevity. $B$ represents the base point of the curve and $i$ a little-endian 4-byte string.

By truncating $Z$ to 28 bytes and then multiplying by 8, the authors ensure that the scalar is a multiple of 8 and that bit 255 is 0. When $k{_L}^P$ is added, it contributes a 1 on bit 254.

Note that as part of the initial conversion from seed to master key $\tilde{k}$, only those $\tilde{k}$ that produce a $k_L$ with bit 253 set to 0 are admitted. This prevents overflow when adding, which would affect bit 254.

As a result, the children are guaranteed to be safe in accordance with the requirements of Ed25519, for many levels of derivation.

To understand how many levels can be safely performed, consider that the root key $a$ can be expressed as follows: $a = 2^{254} + 8b$, where $b < 2^{250}$, due to the bit manipulation and the way it is generated. 

Thus, $a = 2^{254} + 8b < 2^{254} + 2^{253}$.

At the next level, $a_{1} = a + 8[Z_L] < 2^{254} + 2^{253} + 8 \cdot 2^{224} = 2^{254} + 2^{253} + 2^{227}$.

At the next level, $a_{2} = a_{1} + 8[Z_L] < 2^{254} + 2^{253} + 2^{227} + 2^{227} = 2^{254} + 2^{253} + 2^{228}$.

At the next level, $a_{3} = a_{2} + 8[Z_L] < 2^{254} + 2^{253} + 2^{228} + 2^{227}$.

In short, the key at level $D$ depth (called $j$ in the paper) is defined by the inequality, $a_D < a + D\cdot2^{227} = 2^{254} + 2^{253} + D\cdot2^{227}$.

The paper specifies $D \leq 2^{20}$, allowing for a maximum 1048576 levels of derivation, such that $a_{D} \leq 2^{254} + 2^{253} + 2^{227 + 20 = 247}$ - a form suitable for our purposes.

Unfortunately, offering this many levels comes at a cost: only 28 bytes (224 bits) of randomization takes place for every level. 

Consider the scenario where a single child private key is compromised. The security of its parent key would degrade to around only $\sqrt{2^{224}}=2^{112}$ due to the Birthday Attack, below the 128 bits of security offered by Ed25519. This contradicts the author's claims that no loss of security occurs as a result of their method.

In other words, computing the parent key may become more than 64,000 times ($\sqrt{2^{32}}=2^{16}$) easier for the attacker as a result of doing away with 4 bytes of entropy.

If many sibling private keys are compromised, the parent key's security is degraded even further. And the same applies to further ancestors, up to the root private key.

\section{Improving the BIP32-Ed25519 Standard}

The purpose of our modifications is to make derived keys less related, so that there is less security degradation if some private keys are compromised.

Define $g$, such that $Z_L = Z_{[:256-g]}$. The BIP32-Ed25519 paper~\cite{BIP32-Ed25519} set $g=32$ to arrive on 28 bytes.

An Ed25519 scalar has 251 bits available for randomization (255, 254, 2, 1, 0 are fixed), with an additional bit (253) having to be 0 to avoid overflow. The level at which we can no longer guarantee 253 will be 0 forms an upper bound.

Recall from \cref{subsec:the_28_left_bytes_of_z} that each $Z_L < 2^{256 - g}$ and is multiplied by 8. Root $a$ must also be less than $2^{254} + 2^{253}$. Define $d = log_2(D)$. The previous paragraph can thus be expressed as follows:
\begin{align*}
  a_D &< a + 8\cdot D\cdot2^{256 - g} \\
  a_D &< a + D\cdot2^{259 - g} \\
  a_D &< a + 2^{259 + d - g}\\
  a_D < 2^{254} +& 2^{253} + 2^{259 + d - g} < 2^{254} + 2^{254}
\end{align*}
The relationship between $d$ and $g$ can be stated as:
\begin{align*}
  253 &> 259 + d - g \\
  &\Downarrow \\
  g &> d + 6
\end{align*}

 $g = d + 6$ defines the first level at which the 3rd highest bit overflows into the 2nd highest bit, which is set to 1 already and would overflow into the highest bit. $d$ must necessarily be $d < g - 6$.

 As a result, the number of safe levels $S$ of derivation that can be performed is one level fewer than $D = 2^{d} < 2^{g - 6}$:

 \begin{equation*}
  S = D - 1 = 2^{d} - 1
  \end{equation*}

\begin{table}[h]
  \centering
  \begin{tabular}{|c|c|c|c|}
  \hline
  g & d  & D & S  \\
  \hline
  8 & 2 & 4 & 3 \\
  9 & 3 & 8 & 7 \\
  16 & 10 & 1024 & 1023 \\
  32 & 26 & 67108864 & 67108863  \\
  \hline
  \end{tabular}
  \caption{Bits to randomize $g$ resulting in levels $D$. $S$ is the number of safe levels.}
  \label{tab:g_and_d}
\end{table}


\Cref{tab:g_and_d} presents a selection of possible values.
$g=8$ is the most secure and breaks at the byte boundary (256 bits - 8 bits = 31 bytes) but it results in only 3 levels of derivation - too few to satisfy the derivation schemes mentioned in \cref{subsec:hd_wallets}, which require 5.

$g=16$ is the next value that breaks at the byte boundary (equal to 30 bytes) and provides more than enough levels. It randomizes 240 out of the 251 bits available in an Ed25519 scalar.

$g=9$ is the most secure value that also allows for enough derivation levels, allowing for 7 levels of derivation. It randomizes 246 out of the 251 bits available in a Ed25519 scalar. The only downside is that it does not break at the byte boundary, requiring a bit of extra care when implementing.

It also becomes apparent that the recommended maximum level of depth in the BIP32-Ed25519 paper, $2^{20}$, is lower than the actual number of levels that can be performed without bit overflows breaking the scalar's Ed25519-compatibility: $2^{26} - 1$.

Indeed, for $D=2^{20}$, $a_{D} \leq 2^{254} + 2^{253} + 2^{247} < 2^{254} + 2^{253} + 2^{227+26=253} = 2^{255}$.


\section{Implementations}


The suggestions provided in this paper have been implemented as part of the ARC52 Bip32-Ed25519 libraries provided by the Algorand Foundation~\cite{AF}.

\begin{itemize}
  \item \href{https://github.com/algorandfoundation/bip32-ed25519-kotlin}{Kotlin}
  \item \href{https://github.com/algorandfoundation/bip32-ed25519-swift}{Swift}
  \item \href{https://github.com/ehanoc/ARCs/tree/wallet-api-context/assets/arc-0052}{TypeScript}
\end{itemize}


\section{Conclusion}

In this paper a modification to the BIP32-Ed25519 standard has been presented that provides a more secure way to derive Ed25519 keys, at the cost of reducing the number of levels of child keys. We have shown that this modification provides a significant improvement in security, while still providing a sufficient number of child keys for most applications.

In particular, setting $g=9$ allows for the most secure implementation that also satisfies the requirements of the most common derivation schemes.

We have also provided implementations of this modification in several programming languages. We hope that this modification will be adopted by the wider community and help to improve the security of Ed25519-based hierarchical deterministic wallets.

%%%%%%%%%%%%%%
% References %
%%%%%%%%%%%%%%

\nocite{*}
\bibliographystyle{plain}
\bibliography{sources}

\end{document}

%%% Local Variables:
%%% mode: latex
%%% TeX-master: t
%%% End:
